\documentclass[journal]{IEEEtran}

\usepackage{tabularx}
\usepackage{varioref}
\usepackage{subfigure}
\usepackage{color}

\usepackage{xspace}
\usepackage{amsmath}
\usepackage{capt-of}
\usepackage{caption}
\usepackage{listings}
\usepackage{array}
\usepackage{ragged2e}
\usepackage{booktabs}
\usepackage{float}
\usepackage{colortbl}
\usepackage{algorithm2e}

\newcommand{\PreserveBackslash}[1]{\let\temp=\\#1\let\\=\temp}
\newcolumntype{L}[1]{>{\PreserveBackslash\RaggedRight}p{#1}}
\newcolumntype{R}[1]{>{\PreserveBackslash\RaggedLeft}p{#1}}

\definecolor{orange}{rgb}{1,0.5,0}

\ifx\pdfoutput\undefined
\usepackage{graphicx}
\else
\usepackage[pdftex]{graphicx}
\fi

\hyphenation{op-tical net-works semi-conduc-tor}


\begin{document}
\newcommand{\ra}[1]{\renewcommand{\arraystretch}{#1}}

\title{A Group Mobility Model Generator for MANETs based on Terrain Maps}

\author{Hagen~Paul~Pfeifer\\Furtwangen University\\Faculty of Business
Information Systems\\hagen.pfeifer@hs-furtwangen.de}



\markboth{\today{} --- Munich}{}
\maketitle


\begin{abstract}
A genetic algorithm is chosen to calculate the ideal configuration for every
Condition Class. All kind of calculated intervals and hold-times are guarded
through pre-defined upper and lower thresholds guarantee a complaisant protocol
behavior.
\end{abstract}

% 
\begin{keywords}
Genetic algorithms, Software performance, Functional analysis
\end{keywords}
% 
\IEEEpeerreviewmaketitle


%%%%%%%%%%%%%%%%%%%%%%%%%%%%%%%%%%%%%%%
\section{Introduction}

%%%%%%%%%%%%%%%%%%%%%%%%%%%%%%%%%%%%%%%
\section{Functioning}

%%%%%%%%%%%
\subsection{Group Leader}

Gravity Penalty

%%%%%%%%%%%%%%%%%%%%
\section{Conclusion}


\nocite{*}
\bibliography{literature}
\bibliographystyle{plain}


\end{document}
